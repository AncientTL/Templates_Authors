%%JOURNAL TEMPLATE ANCIENT TL
%%author: Regina DeWitt, Sebastian Kreutzer
%%licence: LaTeX project public license (LPPL) in version 1.3c or later
%%version: 2025-04-01

\documentclass{article}
\usepackage[preprint]{config/ATL} %% allowed flags: preprint, proof, final

%%=========================================================%%
%%CONFIG - TO BE MODIFIED BY COPY-EDITOR ONLY
%%=========================================================%%
\setcounter{page}{1} 					             %page start counter
\AuthorHeaderATL{Author}	                         %family name of first author 
\VolHeaderATL{X}					                 %ATL Vol
\NoHeaderATL{X}					                     %ATL No
\YearHeaderATL{X}				                     %ATL year

\lstset{language=R} %% define programming language for listing package.
%%=========================================================%%
%%PAPER TITLE 
%%=========================================================%%
\begin{document}
%%
%% 
%% MANUSCRIPT TI1TLE
\title{A Fairy Tale on Luminescence and ESR Dating}
%%
%% AUTHOR(S) AND AFFILATION AND CONTRIBUTION HISTORY
\author{Electra Trap \orcid{0000-0002-9079-593X},$^{1\ast}$\\
\\
\normalsize{$^{1}$ Centre for Luminescence and ESR dating, Ancient TL}\\
\\
\normalsize{$^\ast$Corresponding Author: electra.trap@ancient.tl}\\
\\
%\it\normalsize{Received: June 25, 2014; in final form: December 12, 2014}\\
}

%%=========================================================%%
%%TODO LIST FOR AUTHOR - JUST UNCOMMENT FOR USAGE
%%=========================================================%%
%\makeatother
%\listoftodos 
%\relax
%
%%%%%TODO FOR AUTHOR
%\todo{ToDo}

\maketitle
%%=========================================================%%
%%MAINTEXT
%%=========================================================%%

%%%%%%%%% Abstract

\begin{abstract}
Your abstract. Your abstract. Your abstract. Your abstract. Your abstract. Your abstract.
Your abstract. Your abstract. Your abstract. Your abstract. Your abstract. Your abstract.
 \\
   \\
   Keywords: Keyword 1, Keyword 2, Keyword 3, Keyword 4, Keyword 5
\end{abstract}


%%%%%%%%%  BODY TEXT
%-------------------------------------------------------------------------
\section{Formatting the document}
%-------------------------------------------------------------------------
\subsection{References}
References in Ancient TL \citep{ATL-website} are inserted using standard \LaTeX citations. 
Example for the the article radiation damage as a research tool for geology and prehistory: \cite{Grogler1958fk}

%-------------------------------------------------------------------------
\subsection{Illustrations, graphs}

Resize fonts in figures to match the font in the body text, and choose line 
widths which render effectively in print.  Many readers (and reviewers), 
even of an electronic copy, will choose to print your paper in order to read it.  You cannot
insist that they do otherwise, and therefore must not assume that they can
zoom in to see tiny details on a graphic.

When placing figures in \LaTeX, it's almost always best to use
\verb+\includegraphics+, and to specify the  figure width as a multiple of
the line width as in the example below

{\small\begin{verbatim}
   \usepackage[dvips]{graphicx} ...
   \includegraphics[width=0.8\linewidth]
                   {myfile.eps}
\end{verbatim}
}


%-------------------------------------------------------------------------
\subsection{Tables}

\begin{table}[h]
\begin{center}
\begin{tabular}{lc}
\hline
Method & Frobnability \\
\hline\hline
Theirs & Frumpy \\
Yours & Frobbly \\
Ours & Makes one's heart Frob\\
\hline
\end{tabular}
\end{center}
\caption{Results.   Ours is better.}
\end{table}


%-------------------------------------------------------------------------
\subsection{Equations}

Equations can be set in two ways: (a) inline ($y = e^x$) and (b) as separate equation:
\begin{equation} 
I(x) = I_{0}(1-exp(\frac{x-a}{b}))
\end{equation}
    
 
%-------------------------------------------------------------------------
\subsection{Footnotes}

Please use footnotes\footnote {This is what a footnote looks like.  It
often distracts the reader from the main flow of the argument.} sparingly.
Indeed, try to avoid footnotes altogether and include necessary peripheral
observations in
the text (within parentheses, if you prefer, as in this sentence).  If you
wish to use a footnote, place it at the bottom of the column on the page on
which it is referenced. 

%-------------------------------------------------------------------------
\subsection{Source code}

Program code/source code can be included using the listings environment, e.g.

\begin{lstlisting}[caption = A simple for-loop]
  for (i in 1:100){
  
    print(''Hello World'')
  
  }
  
\end{lstlisting}



\vspace*{20 pt}

\section*{Acknowledgments} 
We are grateful to the ghost of luminescence.


%%=========================================================%%
%%BELOW = TO BE MODIFIED BY COPY-EDITOR ONLY
%%=========================================================%%
\vspace*{16pt}

{\small
\bibliographystyle{config/icml2013-RD}
\bibliography{main.bib}
}

%\vspace*{16pt}
%\subsection*{Reviewer}
%Name 
\end{document}

